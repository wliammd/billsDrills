\PassOptionsToPackage{unicode=true}{hyperref} % options for packages loaded elsewhere
\PassOptionsToPackage{hyphens}{url}
%
\documentclass[]{book}
\usepackage{lmodern}
\usepackage{amssymb,amsmath}
\usepackage{ifxetex,ifluatex}
\usepackage{fixltx2e} % provides \textsubscript
\ifnum 0\ifxetex 1\fi\ifluatex 1\fi=0 % if pdftex
  \usepackage[T1]{fontenc}
  \usepackage[utf8]{inputenc}
  \usepackage{textcomp} % provides euro and other symbols
\else % if luatex or xelatex
  \usepackage{unicode-math}
  \defaultfontfeatures{Ligatures=TeX,Scale=MatchLowercase}
\fi
% use upquote if available, for straight quotes in verbatim environments
\IfFileExists{upquote.sty}{\usepackage{upquote}}{}
% use microtype if available
\IfFileExists{microtype.sty}{%
\usepackage[]{microtype}
\UseMicrotypeSet[protrusion]{basicmath} % disable protrusion for tt fonts
}{}
\IfFileExists{parskip.sty}{%
\usepackage{parskip}
}{% else
\setlength{\parindent}{0pt}
\setlength{\parskip}{6pt plus 2pt minus 1pt}
}
\usepackage{hyperref}
\hypersetup{
            pdftitle={Bill's Drills Book},
            pdfauthor={William McDonald},
            pdfborder={0 0 0},
            breaklinks=true}
\urlstyle{same}  % don't use monospace font for urls
\usepackage{color}
\usepackage{fancyvrb}
\newcommand{\VerbBar}{|}
\newcommand{\VERB}{\Verb[commandchars=\\\{\}]}
\DefineVerbatimEnvironment{Highlighting}{Verbatim}{commandchars=\\\{\}}
% Add ',fontsize=\small' for more characters per line
\usepackage{framed}
\definecolor{shadecolor}{RGB}{248,248,248}
\newenvironment{Shaded}{\begin{snugshade}}{\end{snugshade}}
\newcommand{\AlertTok}[1]{\textcolor[rgb]{0.94,0.16,0.16}{#1}}
\newcommand{\AnnotationTok}[1]{\textcolor[rgb]{0.56,0.35,0.01}{\textbf{\textit{#1}}}}
\newcommand{\AttributeTok}[1]{\textcolor[rgb]{0.77,0.63,0.00}{#1}}
\newcommand{\BaseNTok}[1]{\textcolor[rgb]{0.00,0.00,0.81}{#1}}
\newcommand{\BuiltInTok}[1]{#1}
\newcommand{\CharTok}[1]{\textcolor[rgb]{0.31,0.60,0.02}{#1}}
\newcommand{\CommentTok}[1]{\textcolor[rgb]{0.56,0.35,0.01}{\textit{#1}}}
\newcommand{\CommentVarTok}[1]{\textcolor[rgb]{0.56,0.35,0.01}{\textbf{\textit{#1}}}}
\newcommand{\ConstantTok}[1]{\textcolor[rgb]{0.00,0.00,0.00}{#1}}
\newcommand{\ControlFlowTok}[1]{\textcolor[rgb]{0.13,0.29,0.53}{\textbf{#1}}}
\newcommand{\DataTypeTok}[1]{\textcolor[rgb]{0.13,0.29,0.53}{#1}}
\newcommand{\DecValTok}[1]{\textcolor[rgb]{0.00,0.00,0.81}{#1}}
\newcommand{\DocumentationTok}[1]{\textcolor[rgb]{0.56,0.35,0.01}{\textbf{\textit{#1}}}}
\newcommand{\ErrorTok}[1]{\textcolor[rgb]{0.64,0.00,0.00}{\textbf{#1}}}
\newcommand{\ExtensionTok}[1]{#1}
\newcommand{\FloatTok}[1]{\textcolor[rgb]{0.00,0.00,0.81}{#1}}
\newcommand{\FunctionTok}[1]{\textcolor[rgb]{0.00,0.00,0.00}{#1}}
\newcommand{\ImportTok}[1]{#1}
\newcommand{\InformationTok}[1]{\textcolor[rgb]{0.56,0.35,0.01}{\textbf{\textit{#1}}}}
\newcommand{\KeywordTok}[1]{\textcolor[rgb]{0.13,0.29,0.53}{\textbf{#1}}}
\newcommand{\NormalTok}[1]{#1}
\newcommand{\OperatorTok}[1]{\textcolor[rgb]{0.81,0.36,0.00}{\textbf{#1}}}
\newcommand{\OtherTok}[1]{\textcolor[rgb]{0.56,0.35,0.01}{#1}}
\newcommand{\PreprocessorTok}[1]{\textcolor[rgb]{0.56,0.35,0.01}{\textit{#1}}}
\newcommand{\RegionMarkerTok}[1]{#1}
\newcommand{\SpecialCharTok}[1]{\textcolor[rgb]{0.00,0.00,0.00}{#1}}
\newcommand{\SpecialStringTok}[1]{\textcolor[rgb]{0.31,0.60,0.02}{#1}}
\newcommand{\StringTok}[1]{\textcolor[rgb]{0.31,0.60,0.02}{#1}}
\newcommand{\VariableTok}[1]{\textcolor[rgb]{0.00,0.00,0.00}{#1}}
\newcommand{\VerbatimStringTok}[1]{\textcolor[rgb]{0.31,0.60,0.02}{#1}}
\newcommand{\WarningTok}[1]{\textcolor[rgb]{0.56,0.35,0.01}{\textbf{\textit{#1}}}}
\usepackage{longtable,booktabs}
% Fix footnotes in tables (requires footnote package)
\IfFileExists{footnote.sty}{\usepackage{footnote}\makesavenoteenv{longtable}}{}
\usepackage{graphicx,grffile}
\makeatletter
\def\maxwidth{\ifdim\Gin@nat@width>\linewidth\linewidth\else\Gin@nat@width\fi}
\def\maxheight{\ifdim\Gin@nat@height>\textheight\textheight\else\Gin@nat@height\fi}
\makeatother
% Scale images if necessary, so that they will not overflow the page
% margins by default, and it is still possible to overwrite the defaults
% using explicit options in \includegraphics[width, height, ...]{}
\setkeys{Gin}{width=\maxwidth,height=\maxheight,keepaspectratio}
\setlength{\emergencystretch}{3em}  % prevent overfull lines
\providecommand{\tightlist}{%
  \setlength{\itemsep}{0pt}\setlength{\parskip}{0pt}}
\setcounter{secnumdepth}{5}
% Redefines (sub)paragraphs to behave more like sections
\ifx\paragraph\undefined\else
\let\oldparagraph\paragraph
\renewcommand{\paragraph}[1]{\oldparagraph{#1}\mbox{}}
\fi
\ifx\subparagraph\undefined\else
\let\oldsubparagraph\subparagraph
\renewcommand{\subparagraph}[1]{\oldsubparagraph{#1}\mbox{}}
\fi

% set default figure placement to htbp
\makeatletter
\def\fps@figure{htbp}
\makeatother

\usepackage{booktabs}
\usepackage{amsthm}
\makeatletter
\def\thm@space@setup{%
  \thm@preskip=8pt plus 2pt minus 4pt
  \thm@postskip=\thm@preskip
}
\makeatother
\usepackage[]{natbib}
\bibliographystyle{apalike}

\title{Bill's Drills Book}
\author{William McDonald}
\date{2020-04-12}

\begin{document}
\maketitle

{
\setcounter{tocdepth}{1}
\tableofcontents
}
\hypertarget{intro}{%
\chapter{Drills: Part of Every Healthy Intellectual Diet}\label{intro}}

The goal of this book is to organize my R drills into reasonable chunks, the better to understand my strengths and weaknesses, and to plan new forays into data science.

Notes on \textbf{bookdown}:

\begin{itemize}
\tightlist
\item
  the \texttt{\_bookdown.yml} file contains a snippet that is important to inserting the word ``Chapter'' before the chapter number in each of the Rmd files.
\item
  packages are indicated in bold, like \textbf{dplyr}
\item
  inline code and filenames are indicated in typerwriter face using backticks, like \texttt{\_bookdown.yml}
\item
  \texttt{\_output.yml} is modified from that used by Xie in his \texttt{bookdown-demo} \citep{R-bookdown}; it evokes \texttt{style.css}, \texttt{toc.css}, \texttt{preamble.tex}, which are also borrowed from Xie.
\end{itemize}

\hypertarget{dataexploration}{%
\chapter{Data Exploration}\label{dataexploration}}

Data exploration is one of the most important aspects of data science and forms the cornerstone of my drills. Nonetheless, I have lots of room for improvement.

I like Hadely Wickham's writing and find his approach exceptionally clear. Therefore, I'll use the \emph{tidyverse}.

\begin{Shaded}
\begin{Highlighting}[]
\KeywordTok{library}\NormalTok{(tidyverse)}
\end{Highlighting}
\end{Shaded}

\hypertarget{counting-things.-the-naming-of-parts.}{%
\section{Counting things. The naming of parts.}\label{counting-things.-the-naming-of-parts.}}

\begin{Shaded}
\begin{Highlighting}[]
\NormalTok{starwars }\OperatorTok\StringTok{ }
\StringTok{  }\KeywordTok{filter}\NormalTok{(}\OperatorTok{!}\KeywordTok{is.na}\NormalTok{(species)) }\OperatorTok\StringTok{ }
\StringTok{  }\KeywordTok{count}\NormalTok{(}\DataTypeTok{species =} \KeywordTok{fct_lump}\NormalTok{(species, }\DecValTok{5}\NormalTok{), }\DataTypeTok{sort =} \OtherTok{TRUE}\NormalTok{) }\OperatorTok\StringTok{ }
\StringTok{  }\KeywordTok{mutate}\NormalTok{(}\DataTypeTok{species =} \KeywordTok{fct_reorder}\NormalTok{(species, n)) }\OperatorTok\StringTok{ }
\StringTok{  }\KeywordTok{ggplot}\NormalTok{(}\KeywordTok{aes}\NormalTok{(species, n)) }\OperatorTok{+}\StringTok{ }
\StringTok{  }\KeywordTok{geom_col}\NormalTok{() }\OperatorTok{+}\StringTok{ }\KeywordTok{coord_flip}\NormalTok{()}
\end{Highlighting}
\end{Shaded}

\begin{figure}
\centering
\includegraphics{test-book_files/figure-latex/starfig-1-1.pdf}
\caption{\label{fig:starfig-1}Starwars Figure 1}
\end{figure}

I like stacked bars for their economy, but it's easy to over do it. Supperimposing gender onto the columns seems easy\ldots{}

\begin{Shaded}
\begin{Highlighting}[]
\NormalTok{starwars }\OperatorTok\StringTok{ }
\StringTok{  }\KeywordTok{filter}\NormalTok{(}\OperatorTok{!}\KeywordTok{is.na}\NormalTok{(species)) }\OperatorTok\StringTok{ }
\StringTok{  }\KeywordTok{count}\NormalTok{(}\DataTypeTok{species =} \KeywordTok{fct_lump}\NormalTok{(species, }\DecValTok{5}\NormalTok{), }\DataTypeTok{gender =} \KeywordTok{fct_lump}\NormalTok{(gender, }\DecValTok{2}\NormalTok{), }\DataTypeTok{sort =} \OtherTok{TRUE}\NormalTok{) }\OperatorTok\StringTok{ }
\StringTok{  }\KeywordTok{mutate}\NormalTok{(}\DataTypeTok{species =} \KeywordTok{fct_reorder}\NormalTok{(species, n)) }\OperatorTok\StringTok{ }
\StringTok{  }\KeywordTok{ggplot}\NormalTok{(}\KeywordTok{aes}\NormalTok{(species, n, }\DataTypeTok{fill =}\NormalTok{ gender)) }\OperatorTok{+}\StringTok{ }
\StringTok{  }\KeywordTok{geom_col}\NormalTok{() }\OperatorTok{+}\StringTok{ }\KeywordTok{coord_flip}\NormalTok{()}
\end{Highlighting}
\end{Shaded}

\begin{verbatim}
## Warning: Factor `gender` contains implicit NA, consider using
## `forcats::fct_explicit_na`
\end{verbatim}

\includegraphics{test-book_files/figure-latex/unnamed-chunk-2-1.pdf}

But note that I've got a problem: the Droids, which outnumber the Gungans, are now reordered to \emph{after} the Gungans. This happens because the \(n\) that we're counting comprises subcategories of species \emph{and} gender. Only three Gungan males exist (and no females), but that is enough to tie the Droid NA category. The Droid NA category come after the Gungan category, presumably because \emph{male} comes before \emph{NA}, or because NA comes last (more likely).

Exploring this, I see that I'm getting warning messages about the implicit NA's in gender. Note that the following renders a slightly different plot. I \emph{still} have not fixed the order of the species.

\begin{Shaded}
\begin{Highlighting}[]
\NormalTok{starwars }\OperatorTok\StringTok{ }
\StringTok{  }\KeywordTok{filter}\NormalTok{(}\OperatorTok{!}\KeywordTok{is.na}\NormalTok{(species)) }\OperatorTok\StringTok{ }
\StringTok{  }\KeywordTok{count}\NormalTok{(}\DataTypeTok{species =} \KeywordTok{fct_lump}\NormalTok{(species, }\DecValTok{5}\NormalTok{), }\DataTypeTok{gender =} \KeywordTok{fct_lump}\NormalTok{(gender, }\DecValTok{2}\NormalTok{), }\DataTypeTok{sort =} \OtherTok{TRUE}\NormalTok{) }\OperatorTok\StringTok{ }
\StringTok{  }\KeywordTok{mutate}\NormalTok{(}\DataTypeTok{gender =} \KeywordTok{fct_explicit_na}\NormalTok{(gender),}
         \DataTypeTok{species =} \KeywordTok{fct_reorder}\NormalTok{(species, n)) }\OperatorTok\StringTok{ }
\StringTok{  }\KeywordTok{ggplot}\NormalTok{(}\KeywordTok{aes}\NormalTok{(species, n, }\DataTypeTok{fill =}\NormalTok{ gender)) }\OperatorTok{+}\StringTok{ }
\StringTok{  }\KeywordTok{geom_col}\NormalTok{() }\OperatorTok{+}\StringTok{ }\KeywordTok{coord_flip}\NormalTok{()}
\end{Highlighting}
\end{Shaded}

\begin{verbatim}
## Warning: Factor `gender` contains implicit NA, consider using
## `forcats::fct_explicit_na`
\end{verbatim}

\includegraphics{test-book_files/figure-latex/unnamed-chunk-3-1.pdf}

The trick here is to use \texttt{group\_by()} and \texttt{ungroup()} wisely.

\begin{Shaded}
\begin{Highlighting}[]
\NormalTok{starwars }\OperatorTok\StringTok{ }\KeywordTok{filter}\NormalTok{(}\OperatorTok{!}\KeywordTok{is.na}\NormalTok{(species)) }\OperatorTok\StringTok{ }
\StringTok{  }\KeywordTok{mutate}\NormalTok{(}\DataTypeTok{species =} \KeywordTok{fct_lump}\NormalTok{(species, }\DecValTok{5}\NormalTok{)) }\OperatorTok\StringTok{ }
\StringTok{  }\KeywordTok{group_by}\NormalTok{(species) }\OperatorTok\StringTok{ }
\StringTok{  }\KeywordTok{mutate}\NormalTok{(}\DataTypeTok{typeCount =} \KeywordTok{n}\NormalTok{()) }\OperatorTok\StringTok{ }
\StringTok{  }\KeywordTok{ungroup}\NormalTok{() }\OperatorTok\StringTok{ }
\StringTok{  }\KeywordTok{mutate}\NormalTok{(}\DataTypeTok{species =} \KeywordTok{fct_reorder}\NormalTok{(species, typeCount)) }\OperatorTok\StringTok{ }
\StringTok{  }\KeywordTok{ggplot}\NormalTok{()}\OperatorTok{+}
\StringTok{  }\KeywordTok{geom_bar}\NormalTok{(}\KeywordTok{aes}\NormalTok{(species, }\DataTypeTok{fill =}\NormalTok{ gender))}\OperatorTok{+}
\StringTok{  }\KeywordTok{coord_flip}\NormalTok{()}
\end{Highlighting}
\end{Shaded}

\includegraphics{test-book_files/figure-latex/unnamed-chunk-4-1.pdf}

As opposed to using \texttt{count()}, which progressively narrows the information available to be used, by using \texttt{group\_by()}/\texttt{mutate()}/\texttt{ungroup()} with \texttt{geom\_bar()} we have all of the variables still available for plotting.

\hypertarget{summarize-is-another-very-useful-function}{%
\section{Summarize is another very useful function:}\label{summarize-is-another-very-useful-function}}

\begin{Shaded}
\begin{Highlighting}[]
\NormalTok{starwars }\OperatorTok\StringTok{ }
\StringTok{  }\KeywordTok{filter}\NormalTok{(}\OperatorTok{!}\NormalTok{(}\KeywordTok{is.na}\NormalTok{(species))) }\OperatorTok\StringTok{ }
\StringTok{  }\KeywordTok{group_by}\NormalTok{(species) }\OperatorTok\StringTok{ }
\StringTok{  }\KeywordTok{summarize}\NormalTok{(}\DataTypeTok{n=}\KeywordTok{n}\NormalTok{(), }\DataTypeTok{mean =} \KeywordTok{mean}\NormalTok{(height, }\DataTypeTok{na.rm =} \OtherTok{TRUE}\NormalTok{)) }\OperatorTok\StringTok{ }
\StringTok{  }\KeywordTok{arrange}\NormalTok{(}\KeywordTok{desc}\NormalTok{(n))}
\end{Highlighting}
\end{Shaded}

\begin{verbatim}
## # A tibble: 37 x 3
##    species      n  mean
##    <chr>    <int> <dbl>
##  1 Human       35  177.
##  2 Droid        5  140 
##  3 Gungan       3  209.
##  4 Kaminoan     2  221 
##  5 Mirialan     2  168 
##  6 Twi'lek      2  179 
##  7 Wookiee      2  231 
##  8 Zabrak       2  173 
##  9 Aleena       1   79 
## 10 Besalisk     1  198 
## # ... with 27 more rows
\end{verbatim}

\hypertarget{referencing-other-parts-of-the-document}{%
\section{Referencing other parts of the document}\label{referencing-other-parts-of-the-document}}

This is a good place to practice referencing figures. Say that I want to refer the reader back to my first starwars figure. See Figure \ref{fig:starfig-1}.

I can reference other pages in a similar fashion. See Chapter \ref{subset}. Note that this works by referencing a \{\#label\} placed in the chapter title.

See Chapter \ref{intro}

See Chapter \ref{dataexploration}

Note that the \{\#label\} uses a single run-together word. It does not tolerate spaces and this cannot be overcome by `quoting' it.

\hypertarget{referencing-citations}{%
\section{Referencing citations:}\label{referencing-citations}}

In order to insert citations, one needs a .bib file in the project. I've included one in this project as \emph{book.bib}. The yml header in Chapter \ref{intro} needs to have a \(bibliography:\) and \(biblio-style:\) line added.

To insert a citation, use the \textbf{citr} Addin from RStudio. \textbf{bookdown}, for instance, is cited thusly \citep{R-bookdown}. Note that I need to figure out an adequate workflow of references. The convenience of Endnote in MS Word will not be available. Nonetheless, if I populate the book.bib and packages.bib files carefully, with .txt files generated in Endnote, I should be OK.

For instance, a recent dump of my Endnote library is in bookFromEndnote.txt. This can be opened in RStudio, and I can copy-and-paste references from the .txt file to my book.bib. For instance, if I have a breast paper that I want to cite here \citep{RN2750}, I'd copy-and-paste the reference from bookFromEndnot.text to book.bib.

Of note, Yihui Xie includes a nifty bit of code to automatically generate a bib database for R packages:

\begin{Shaded}
\begin{Highlighting}[]
\NormalTok{knitr}\OperatorTok{::}\KeywordTok{write_bib}\NormalTok{(}\KeywordTok{c}\NormalTok{(}\KeywordTok{.packages}\NormalTok{(), }\StringTok{'bookdown'}\NormalTok{, }\StringTok{'knitr'}\NormalTok{, }\StringTok{'rmarkdown'}\NormalTok{, }\StringTok{'tidyverse'}\NormalTok{, }\StringTok{'ComplexHeatmap'}\NormalTok{), }\StringTok{'packages.bib'}\NormalTok{)}
\end{Highlighting}
\end{Shaded}

References appear automatically at the end of a chapter.

\hypertarget{sampling}{%
\chapter{Sampling}\label{sampling}}

\hypertarget{think-about-throwing-a-bunch-of-dice.}{%
\section{Think about throwing a bunch of dice.}\label{think-about-throwing-a-bunch-of-dice.}}

\begin{Shaded}
\begin{Highlighting}[]
\KeywordTok{sample}\NormalTok{(}\DecValTok{1}\OperatorTok{:}\DecValTok{6}\NormalTok{, }\DataTypeTok{size=}\DecValTok{100}\NormalTok{, }\DataTypeTok{replace=}\OtherTok{TRUE}\NormalTok{) }
\end{Highlighting}
\end{Shaded}

\begin{verbatim}
##   [1] 1 5 5 2 4 1 3 4 1 2 3 3 6 6 1 1 5 3 3 2 3 4 2 2 3 3 3 4 6 5 4 6 5 1 3 6 5
##  [38] 2 2 1 5 5 5 1 1 2 3 2 3 2 3 1 2 3 1 1 5 2 6 4 2 4 4 4 6 6 6 2 2 5 1 2 3 1
##  [75] 5 5 3 3 2 4 3 6 1 6 1 6 5 5 6 5 3 4 4 5 3 1 3 4 5 2
\end{verbatim}

\begin{Shaded}
\begin{Highlighting}[]
\KeywordTok{sample}\NormalTok{(}\DecValTok{1}\OperatorTok{:}\DecValTok{6}\NormalTok{, }\DataTypeTok{size=}\DecValTok{100}\NormalTok{, }\DataTypeTok{replace=}\OtherTok{TRUE}\NormalTok{) }\OperatorTok\StringTok{ }\KeywordTok{table}\NormalTok{()}
\end{Highlighting}
\end{Shaded}

\begin{verbatim}
## .
##  1  2  3  4  5  6 
## 18 20 20  7 14 21
\end{verbatim}

\begin{Shaded}
\begin{Highlighting}[]
\KeywordTok{sample}\NormalTok{(}\DecValTok{1}\OperatorTok{:}\DecValTok{6}\NormalTok{, }\DataTypeTok{size=}\DecValTok{100}\NormalTok{, }\DataTypeTok{replace=}\OtherTok{TRUE}\NormalTok{) }\OperatorTok\StringTok{ }\KeywordTok{table}\NormalTok{() }\OperatorTok\StringTok{ }\KeywordTok{prop.table}\NormalTok{()}
\end{Highlighting}
\end{Shaded}

\begin{verbatim}
## .
##    1    2    3    4    5    6 
## 0.18 0.20 0.20 0.16 0.12 0.14
\end{verbatim}

\hypertarget{a-keen-way-to-divide-up-a-dataset-into-testing-and-training-components.}{%
\section{A keen way to divide up a dataset into testing and training components.}\label{a-keen-way-to-divide-up-a-dataset-into-testing-and-training-components.}}

\begin{Shaded}
\begin{Highlighting}[]
\NormalTok{x <-}\StringTok{ }\DecValTok{1}\OperatorTok{:}\DecValTok{50}
\NormalTok{y <-}\StringTok{ }\DecValTok{51}\OperatorTok{:}\DecValTok{100}

\NormalTok{df <-}\StringTok{ }\KeywordTok{data.frame}\NormalTok{(x,y)}
\NormalTok{df}
\end{Highlighting}
\end{Shaded}

\begin{verbatim}
##     x   y
## 1   1  51
## 2   2  52
## 3   3  53
## 4   4  54
## 5   5  55
## 6   6  56
## 7   7  57
## 8   8  58
## 9   9  59
## 10 10  60
## 11 11  61
## 12 12  62
## 13 13  63
## 14 14  64
## 15 15  65
## 16 16  66
## 17 17  67
## 18 18  68
## 19 19  69
## 20 20  70
## 21 21  71
## 22 22  72
## 23 23  73
## 24 24  74
## 25 25  75
## 26 26  76
## 27 27  77
## 28 28  78
## 29 29  79
## 30 30  80
## 31 31  81
## 32 32  82
## 33 33  83
## 34 34  84
## 35 35  85
## 36 36  86
## 37 37  87
## 38 38  88
## 39 39  89
## 40 40  90
## 41 41  91
## 42 42  92
## 43 43  93
## 44 44  94
## 45 45  95
## 46 46  96
## 47 47  97
## 48 48  98
## 49 49  99
## 50 50 100
\end{verbatim}

\begin{Shaded}
\begin{Highlighting}[]
\KeywordTok{set.seed}\NormalTok{(}\DecValTok{0}\NormalTok{)}
\NormalTok{train_indexes =}\StringTok{ }\KeywordTok{sample}\NormalTok{(}\DecValTok{1}\OperatorTok{:}\KeywordTok{nrow}\NormalTok{(df), }\FloatTok{.7} \OperatorTok{*}\StringTok{ }\KeywordTok{nrow}\NormalTok{(df))}

\NormalTok{train_set <-}\StringTok{ }\NormalTok{df[train_indexes,]}
\NormalTok{test_set <-}\StringTok{ }\NormalTok{df[}\OperatorTok{-}\NormalTok{train_indexes,]}
\end{Highlighting}
\end{Shaded}

\hypertarget{factor-practice}{%
\chapter{Factor Practice}\label{factor-practice}}

\begin{Shaded}
\begin{Highlighting}[]
\NormalTok{cups <-}\StringTok{ }\KeywordTok{c}\NormalTok{(}\StringTok{"small"}\NormalTok{, }\StringTok{"medium"}\NormalTok{, }\StringTok{"large"}\NormalTok{)}
\NormalTok{manyCups <-}\StringTok{ }\KeywordTok{sample}\NormalTok{(cups, }\DataTypeTok{size =} \DecValTok{100}\NormalTok{, }\DataTypeTok{replace =} \OtherTok{TRUE}\NormalTok{)}
\NormalTok{sizesCups <-}\StringTok{ }\KeywordTok{factor}\NormalTok{(manyCups, }\DataTypeTok{levels =} \KeywordTok{c}\NormalTok{(}\StringTok{"small"}\NormalTok{, }\StringTok{"medium"}\NormalTok{, }\StringTok{"large"}\NormalTok{))}
\NormalTok{sizesCups}
\end{Highlighting}
\end{Shaded}

\begin{verbatim}
##   [1] medium small  large  medium small  small  large  medium medium large 
##  [11] large  medium medium medium medium small  medium medium medium medium
##  [21] small  large  large  medium large  large  medium large  large  small 
##  [31] small  small  small  large  medium large  small  small  medium small 
##  [41] small  small  small  large  medium small  small  large  large  large 
##  [51] medium medium medium large  medium medium large  large  large  small 
##  [61] medium medium small  large  large  medium large  medium small  medium
##  [71] small  large  large  small  medium small  large  medium large  large 
##  [81] small  small  medium medium medium small  small  small  medium small 
##  [91] large  medium large  large  medium large  large  small  small  medium
## Levels: small medium large
\end{verbatim}

\hypertarget{crossing-trial}{%
\chapter{Crossing Trial}\label{crossing-trial}}

From David Robinson birthday paradox Rblogger at \url{https://www.r-bloggers.com/the-birthday-paradox-puzzle-tidy-simulation-in-r/}

\begin{Shaded}
\begin{Highlighting}[]
\NormalTok{summarized <-}\StringTok{ }\KeywordTok{crossing}\NormalTok{(}\DataTypeTok{people =} \KeywordTok{seq}\NormalTok{(}\DecValTok{2}\NormalTok{, }\DecValTok{50}\NormalTok{, }\DecValTok{2}\NormalTok{),}
                       \DataTypeTok{trial =} \DecValTok{1}\OperatorTok{:}\DecValTok{100}\NormalTok{) }\OperatorTok
\StringTok{  }\KeywordTok{mutate}\NormalTok{(}\DataTypeTok{birthday =} \KeywordTok{map}\NormalTok{(people, }\OperatorTok{~}\StringTok{ }\KeywordTok{sample}\NormalTok{(}\DecValTok{365}\NormalTok{, .x, }\DataTypeTok{replace =} \OtherTok{TRUE}\NormalTok{)),}
         \DataTypeTok{multiple =} \KeywordTok{map_lgl}\NormalTok{(birthday, }\OperatorTok{~}\StringTok{ }\KeywordTok{any}\NormalTok{(}\KeywordTok{duplicated}\NormalTok{(.x)))) }\OperatorTok
\StringTok{  }\KeywordTok{group_by}\NormalTok{(people) }\OperatorTok
\StringTok{  }\KeywordTok{summarize}\NormalTok{(}\DataTypeTok{chance =} \KeywordTok{mean}\NormalTok{(multiple))}

\KeywordTok{ggplot}\NormalTok{(summarized, }\KeywordTok{aes}\NormalTok{(people, chance)) }\OperatorTok{+}
\StringTok{  }\KeywordTok{geom_line}\NormalTok{() }\OperatorTok{+}
\StringTok{  }\KeywordTok{scale_y_continuous}\NormalTok{(}\DataTypeTok{labels =}\NormalTok{ scales}\OperatorTok{::}\KeywordTok{percent_format}\NormalTok{()) }\OperatorTok{+}
\StringTok{  }\KeywordTok{labs}\NormalTok{(}\DataTypeTok{y =} \StringTok{"Probability two have the same birthday"}\NormalTok{)}
\end{Highlighting}
\end{Shaded}

\includegraphics{test-book_files/figure-latex/unnamed-chunk-10-1.pdf}

\begin{Shaded}
\begin{Highlighting}[]
\CommentTok{# Checking the work with pbirthday function}
\NormalTok{summarized }\OperatorTok\StringTok{ }
\StringTok{  }\KeywordTok{mutate}\NormalTok{(}\DataTypeTok{exact =} \KeywordTok{map_dbl}\NormalTok{(people, pbirthday)) }\OperatorTok\StringTok{ }
\StringTok{  }\KeywordTok{ggplot}\NormalTok{(}\KeywordTok{aes}\NormalTok{(people, chance)) }\OperatorTok{+}
\StringTok{  }\KeywordTok{geom_line}\NormalTok{() }\OperatorTok{+}
\StringTok{  }\KeywordTok{geom_line}\NormalTok{(}\KeywordTok{aes}\NormalTok{(}\DataTypeTok{y =}\NormalTok{ exact), }\DataTypeTok{lty =} \DecValTok{2}\NormalTok{, }\DataTypeTok{color =} \StringTok{"blue"}\NormalTok{) }\OperatorTok{+}
\StringTok{  }\KeywordTok{scale_y_continuous}\NormalTok{(}\DataTypeTok{labels =}\NormalTok{ scales}\OperatorTok{::}\KeywordTok{percent_format}\NormalTok{()) }\OperatorTok{+}
\StringTok{  }\KeywordTok{labs}\NormalTok{(}\DataTypeTok{y =} \StringTok{"Probability two have the same birthday"}\NormalTok{)}
\end{Highlighting}
\end{Shaded}

\includegraphics{test-book_files/figure-latex/unnamed-chunk-10-2.pdf}

\hypertarget{by-any-other-name}{%
\chapter{By Any Other Name}\label{by-any-other-name}}

This deceptively simple-seeming idea gets complex quickly. The following YouTube was a nice description of the process: \url{https://www.youtube.com/watch?v=Okc0IL5uTnA}

\begin{Shaded}
\begin{Highlighting}[]
\NormalTok{my.data <-}\StringTok{ }\KeywordTok{data.frame}\NormalTok{(}\DataTypeTok{colOne=}\DecValTok{1}\OperatorTok{:}\DecValTok{3}\NormalTok{, }\DataTypeTok{column2=}\DecValTok{4}\OperatorTok{:}\DecValTok{6}\NormalTok{, }\DataTypeTok{column_3=}\DecValTok{7}\OperatorTok{:}\DecValTok{9}\NormalTok{)}
\KeywordTok{rownames}\NormalTok{(my.data) <-}\StringTok{ }\KeywordTok{c}\NormalTok{(}\StringTok{"ant"}\NormalTok{, }\StringTok{"bee"}\NormalTok{, }\StringTok{"cat"}\NormalTok{)}
\KeywordTok{names}\NormalTok{(my.data)}
\end{Highlighting}
\end{Shaded}

\begin{verbatim}
## [1] "colOne"   "column2"  "column_3"
\end{verbatim}

\begin{Shaded}
\begin{Highlighting}[]
\KeywordTok{colnames}\NormalTok{(my.data)}
\end{Highlighting}
\end{Shaded}

\begin{verbatim}
## [1] "colOne"   "column2"  "column_3"
\end{verbatim}

\begin{Shaded}
\begin{Highlighting}[]
\CommentTok{#make some changes}
\KeywordTok{names}\NormalTok{(my.data) <-}\StringTok{ }\KeywordTok{c}\NormalTok{(}\StringTok{"col_1"}\NormalTok{, }\StringTok{"col_2"}\NormalTok{, }\StringTok{"col_3"}\NormalTok{)}
\NormalTok{my.data}
\end{Highlighting}
\end{Shaded}

\begin{verbatim}
##     col_1 col_2 col_3
## ant     1     4     7
## bee     2     5     8
## cat     3     6     9
\end{verbatim}

\begin{Shaded}
\begin{Highlighting}[]
\KeywordTok{names}\NormalTok{(my.data)[}\DecValTok{3}\NormalTok{] <-}\StringTok{ "col.3"}
\NormalTok{my.data}
\end{Highlighting}
\end{Shaded}

\begin{verbatim}
##     col_1 col_2 col.3
## ant     1     4     7
## bee     2     5     8
## cat     3     6     9
\end{verbatim}

\begin{Shaded}
\begin{Highlighting}[]
\KeywordTok{names}\NormalTok{(my.data)[}\KeywordTok{names}\NormalTok{(my.data)}\OperatorTok{==}\StringTok{"col_2"}\NormalTok{]}
\end{Highlighting}
\end{Shaded}

\begin{verbatim}
## [1] "col_2"
\end{verbatim}

\begin{Shaded}
\begin{Highlighting}[]
\NormalTok{my.data[}\StringTok{"col_2"}\NormalTok{]}
\end{Highlighting}
\end{Shaded}

\begin{verbatim}
##     col_2
## ant     4
## bee     5
## cat     6
\end{verbatim}

\begin{Shaded}
\begin{Highlighting}[]
\NormalTok{my.data}\OperatorTok{$}\NormalTok{col_}\DecValTok{2}
\end{Highlighting}
\end{Shaded}

\begin{verbatim}
## [1] 4 5 6
\end{verbatim}

\begin{Shaded}
\begin{Highlighting}[]
\NormalTok{my.data[,}\DecValTok{2}\NormalTok{]}
\end{Highlighting}
\end{Shaded}

\begin{verbatim}
## [1] 4 5 6
\end{verbatim}

\begin{Shaded}
\begin{Highlighting}[]
\KeywordTok{names}\NormalTok{(my.data)[}\KeywordTok{names}\NormalTok{(my.data)}\OperatorTok{==}\StringTok{"col_2"}\NormalTok{] <-}\StringTok{ "col.2"}
\NormalTok{my.data}
\end{Highlighting}
\end{Shaded}

\begin{verbatim}
##     col_1 col.2 col.3
## ant     1     4     7
## bee     2     5     8
## cat     3     6     9
\end{verbatim}

\begin{Shaded}
\begin{Highlighting}[]
\KeywordTok{names}\NormalTok{(my.data) <-}\StringTok{ }\KeywordTok{gsub}\NormalTok{(}\StringTok{"_"}\NormalTok{, }\StringTok{"."}\NormalTok{, }\KeywordTok{names}\NormalTok{(my.data))}
\NormalTok{my.data}
\end{Highlighting}
\end{Shaded}

\begin{verbatim}
##     col.1 col.2 col.3
## ant     1     4     7
## bee     2     5     8
## cat     3     6     9
\end{verbatim}

\begin{Shaded}
\begin{Highlighting}[]
\KeywordTok{rownames}\NormalTok{(my.data)}
\end{Highlighting}
\end{Shaded}

\begin{verbatim}
## [1] "ant" "bee" "cat"
\end{verbatim}

\begin{Shaded}
\begin{Highlighting}[]
\NormalTok{my.data}\OperatorTok{$}\NormalTok{species <-}\StringTok{ }\KeywordTok{rownames}\NormalTok{(my.data)}
\NormalTok{my.data}
\end{Highlighting}
\end{Shaded}

\begin{verbatim}
##     col.1 col.2 col.3 species
## ant     1     4     7     ant
## bee     2     5     8     bee
## cat     3     6     9     cat
\end{verbatim}

\begin{Shaded}
\begin{Highlighting}[]
\KeywordTok{rownames}\NormalTok{(my.data) <-}\StringTok{ }\OtherTok{NULL}
\NormalTok{my.data}
\end{Highlighting}
\end{Shaded}

\begin{verbatim}
##   col.1 col.2 col.3 species
## 1     1     4     7     ant
## 2     2     5     8     bee
## 3     3     6     9     cat
\end{verbatim}

\begin{Shaded}
\begin{Highlighting}[]
\KeywordTok{colnames}\NormalTok{(my.data) <-}\StringTok{ }\KeywordTok{c}\NormalTok{(}\StringTok{"good"}\NormalTok{, }\StringTok{"better"}\NormalTok{, }\StringTok{"best"}\NormalTok{, }\StringTok{"species"}\NormalTok{)}
\NormalTok{my.data}
\end{Highlighting}
\end{Shaded}

\begin{verbatim}
##   good better best species
## 1    1      4    7     ant
## 2    2      5    8     bee
## 3    3      6    9     cat
\end{verbatim}

\begin{Shaded}
\begin{Highlighting}[]
\NormalTok{keep <-}\StringTok{ }\DecValTok{2}\OperatorTok{:}\KeywordTok{ncol}\NormalTok{(my.data)}
\NormalTok{my.data[,keep]}
\end{Highlighting}
\end{Shaded}

\begin{verbatim}
##   better best species
## 1      4    7     ant
## 2      5    8     bee
## 3      6    9     cat
\end{verbatim}

\hypertarget{correlation-plots}{%
\chapter{Correlation Plots}\label{correlation-plots}}

\begin{Shaded}
\begin{Highlighting}[]
\NormalTok{iris}
\end{Highlighting}
\end{Shaded}

\begin{verbatim}
##     Sepal.Length Sepal.Width Petal.Length Petal.Width    Species
## 1            5.1         3.5          1.4         0.2     setosa
## 2            4.9         3.0          1.4         0.2     setosa
## 3            4.7         3.2          1.3         0.2     setosa
## 4            4.6         3.1          1.5         0.2     setosa
## 5            5.0         3.6          1.4         0.2     setosa
## 6            5.4         3.9          1.7         0.4     setosa
## 7            4.6         3.4          1.4         0.3     setosa
## 8            5.0         3.4          1.5         0.2     setosa
## 9            4.4         2.9          1.4         0.2     setosa
## 10           4.9         3.1          1.5         0.1     setosa
## 11           5.4         3.7          1.5         0.2     setosa
## 12           4.8         3.4          1.6         0.2     setosa
## 13           4.8         3.0          1.4         0.1     setosa
## 14           4.3         3.0          1.1         0.1     setosa
## 15           5.8         4.0          1.2         0.2     setosa
## 16           5.7         4.4          1.5         0.4     setosa
## 17           5.4         3.9          1.3         0.4     setosa
## 18           5.1         3.5          1.4         0.3     setosa
## 19           5.7         3.8          1.7         0.3     setosa
## 20           5.1         3.8          1.5         0.3     setosa
## 21           5.4         3.4          1.7         0.2     setosa
## 22           5.1         3.7          1.5         0.4     setosa
## 23           4.6         3.6          1.0         0.2     setosa
## 24           5.1         3.3          1.7         0.5     setosa
## 25           4.8         3.4          1.9         0.2     setosa
## 26           5.0         3.0          1.6         0.2     setosa
## 27           5.0         3.4          1.6         0.4     setosa
## 28           5.2         3.5          1.5         0.2     setosa
## 29           5.2         3.4          1.4         0.2     setosa
## 30           4.7         3.2          1.6         0.2     setosa
## 31           4.8         3.1          1.6         0.2     setosa
## 32           5.4         3.4          1.5         0.4     setosa
## 33           5.2         4.1          1.5         0.1     setosa
## 34           5.5         4.2          1.4         0.2     setosa
## 35           4.9         3.1          1.5         0.2     setosa
## 36           5.0         3.2          1.2         0.2     setosa
## 37           5.5         3.5          1.3         0.2     setosa
## 38           4.9         3.6          1.4         0.1     setosa
## 39           4.4         3.0          1.3         0.2     setosa
## 40           5.1         3.4          1.5         0.2     setosa
## 41           5.0         3.5          1.3         0.3     setosa
## 42           4.5         2.3          1.3         0.3     setosa
## 43           4.4         3.2          1.3         0.2     setosa
## 44           5.0         3.5          1.6         0.6     setosa
## 45           5.1         3.8          1.9         0.4     setosa
## 46           4.8         3.0          1.4         0.3     setosa
## 47           5.1         3.8          1.6         0.2     setosa
## 48           4.6         3.2          1.4         0.2     setosa
## 49           5.3         3.7          1.5         0.2     setosa
## 50           5.0         3.3          1.4         0.2     setosa
## 51           7.0         3.2          4.7         1.4 versicolor
## 52           6.4         3.2          4.5         1.5 versicolor
## 53           6.9         3.1          4.9         1.5 versicolor
## 54           5.5         2.3          4.0         1.3 versicolor
## 55           6.5         2.8          4.6         1.5 versicolor
## 56           5.7         2.8          4.5         1.3 versicolor
## 57           6.3         3.3          4.7         1.6 versicolor
## 58           4.9         2.4          3.3         1.0 versicolor
## 59           6.6         2.9          4.6         1.3 versicolor
## 60           5.2         2.7          3.9         1.4 versicolor
## 61           5.0         2.0          3.5         1.0 versicolor
## 62           5.9         3.0          4.2         1.5 versicolor
## 63           6.0         2.2          4.0         1.0 versicolor
## 64           6.1         2.9          4.7         1.4 versicolor
## 65           5.6         2.9          3.6         1.3 versicolor
## 66           6.7         3.1          4.4         1.4 versicolor
## 67           5.6         3.0          4.5         1.5 versicolor
## 68           5.8         2.7          4.1         1.0 versicolor
## 69           6.2         2.2          4.5         1.5 versicolor
## 70           5.6         2.5          3.9         1.1 versicolor
## 71           5.9         3.2          4.8         1.8 versicolor
## 72           6.1         2.8          4.0         1.3 versicolor
## 73           6.3         2.5          4.9         1.5 versicolor
## 74           6.1         2.8          4.7         1.2 versicolor
## 75           6.4         2.9          4.3         1.3 versicolor
## 76           6.6         3.0          4.4         1.4 versicolor
## 77           6.8         2.8          4.8         1.4 versicolor
## 78           6.7         3.0          5.0         1.7 versicolor
## 79           6.0         2.9          4.5         1.5 versicolor
## 80           5.7         2.6          3.5         1.0 versicolor
## 81           5.5         2.4          3.8         1.1 versicolor
## 82           5.5         2.4          3.7         1.0 versicolor
## 83           5.8         2.7          3.9         1.2 versicolor
## 84           6.0         2.7          5.1         1.6 versicolor
## 85           5.4         3.0          4.5         1.5 versicolor
## 86           6.0         3.4          4.5         1.6 versicolor
## 87           6.7         3.1          4.7         1.5 versicolor
## 88           6.3         2.3          4.4         1.3 versicolor
## 89           5.6         3.0          4.1         1.3 versicolor
## 90           5.5         2.5          4.0         1.3 versicolor
## 91           5.5         2.6          4.4         1.2 versicolor
## 92           6.1         3.0          4.6         1.4 versicolor
## 93           5.8         2.6          4.0         1.2 versicolor
## 94           5.0         2.3          3.3         1.0 versicolor
## 95           5.6         2.7          4.2         1.3 versicolor
## 96           5.7         3.0          4.2         1.2 versicolor
## 97           5.7         2.9          4.2         1.3 versicolor
## 98           6.2         2.9          4.3         1.3 versicolor
## 99           5.1         2.5          3.0         1.1 versicolor
## 100          5.7         2.8          4.1         1.3 versicolor
## 101          6.3         3.3          6.0         2.5  virginica
## 102          5.8         2.7          5.1         1.9  virginica
## 103          7.1         3.0          5.9         2.1  virginica
## 104          6.3         2.9          5.6         1.8  virginica
## 105          6.5         3.0          5.8         2.2  virginica
## 106          7.6         3.0          6.6         2.1  virginica
## 107          4.9         2.5          4.5         1.7  virginica
## 108          7.3         2.9          6.3         1.8  virginica
## 109          6.7         2.5          5.8         1.8  virginica
## 110          7.2         3.6          6.1         2.5  virginica
## 111          6.5         3.2          5.1         2.0  virginica
## 112          6.4         2.7          5.3         1.9  virginica
## 113          6.8         3.0          5.5         2.1  virginica
## 114          5.7         2.5          5.0         2.0  virginica
## 115          5.8         2.8          5.1         2.4  virginica
## 116          6.4         3.2          5.3         2.3  virginica
## 117          6.5         3.0          5.5         1.8  virginica
## 118          7.7         3.8          6.7         2.2  virginica
## 119          7.7         2.6          6.9         2.3  virginica
## 120          6.0         2.2          5.0         1.5  virginica
## 121          6.9         3.2          5.7         2.3  virginica
## 122          5.6         2.8          4.9         2.0  virginica
## 123          7.7         2.8          6.7         2.0  virginica
## 124          6.3         2.7          4.9         1.8  virginica
## 125          6.7         3.3          5.7         2.1  virginica
## 126          7.2         3.2          6.0         1.8  virginica
## 127          6.2         2.8          4.8         1.8  virginica
## 128          6.1         3.0          4.9         1.8  virginica
## 129          6.4         2.8          5.6         2.1  virginica
## 130          7.2         3.0          5.8         1.6  virginica
## 131          7.4         2.8          6.1         1.9  virginica
## 132          7.9         3.8          6.4         2.0  virginica
## 133          6.4         2.8          5.6         2.2  virginica
## 134          6.3         2.8          5.1         1.5  virginica
## 135          6.1         2.6          5.6         1.4  virginica
## 136          7.7         3.0          6.1         2.3  virginica
## 137          6.3         3.4          5.6         2.4  virginica
## 138          6.4         3.1          5.5         1.8  virginica
## 139          6.0         3.0          4.8         1.8  virginica
## 140          6.9         3.1          5.4         2.1  virginica
## 141          6.7         3.1          5.6         2.4  virginica
## 142          6.9         3.1          5.1         2.3  virginica
## 143          5.8         2.7          5.1         1.9  virginica
## 144          6.8         3.2          5.9         2.3  virginica
## 145          6.7         3.3          5.7         2.5  virginica
## 146          6.7         3.0          5.2         2.3  virginica
## 147          6.3         2.5          5.0         1.9  virginica
## 148          6.5         3.0          5.2         2.0  virginica
## 149          6.2         3.4          5.4         2.3  virginica
## 150          5.9         3.0          5.1         1.8  virginica
\end{verbatim}

\begin{Shaded}
\begin{Highlighting}[]
\NormalTok{iris }\OperatorTok\StringTok{ }\KeywordTok{select}\NormalTok{(}\OperatorTok{-}\NormalTok{Species) }\OperatorTok\StringTok{ }\KeywordTok{cor}\NormalTok{()}
\end{Highlighting}
\end{Shaded}

\begin{verbatim}
##              Sepal.Length Sepal.Width Petal.Length Petal.Width
## Sepal.Length    1.0000000  -0.1175698    0.8717538   0.8179411
## Sepal.Width    -0.1175698   1.0000000   -0.4284401  -0.3661259
## Petal.Length    0.8717538  -0.4284401    1.0000000   0.9628654
## Petal.Width     0.8179411  -0.3661259    0.9628654   1.0000000
\end{verbatim}

\begin{Shaded}
\begin{Highlighting}[]
\NormalTok{M <-}\StringTok{ }\NormalTok{iris }\OperatorTok\StringTok{ }\KeywordTok{select}\NormalTok{(}\OperatorTok{-}\NormalTok{Species) }\OperatorTok\StringTok{ }\KeywordTok{cor}\NormalTok{(}\DataTypeTok{method =} \StringTok{"kendall"}\NormalTok{)}
\end{Highlighting}
\end{Shaded}

\begin{Shaded}
\begin{Highlighting}[]
\NormalTok{corrplot}\OperatorTok{::}\KeywordTok{corrplot}\NormalTok{(M)}
\end{Highlighting}
\end{Shaded}

\includegraphics{test-book_files/figure-latex/unnamed-chunk-13-1.pdf}

\begin{Shaded}
\begin{Highlighting}[]
\NormalTok{corrplot}\OperatorTok{::}\KeywordTok{corrplot}\NormalTok{(M, }\DataTypeTok{method =} \StringTok{"color"}\NormalTok{)}
\end{Highlighting}
\end{Shaded}

\includegraphics{test-book_files/figure-latex/unnamed-chunk-13-2.pdf}

\begin{Shaded}
\begin{Highlighting}[]
\NormalTok{corrplot}\OperatorTok{::}\KeywordTok{corrplot}\NormalTok{(M, }\DataTypeTok{method =} \StringTok{"color"}\NormalTok{, }\DataTypeTok{type =} \StringTok{"upper"}\NormalTok{)}
\end{Highlighting}
\end{Shaded}

\includegraphics{test-book_files/figure-latex/unnamed-chunk-13-3.pdf}

\begin{Shaded}
\begin{Highlighting}[]
\NormalTok{corrplot}\OperatorTok{::}\KeywordTok{corrplot}\NormalTok{(M, }\DataTypeTok{method =} \StringTok{"color"}\NormalTok{, }\DataTypeTok{type =} \StringTok{"upper"}\NormalTok{, }\DataTypeTok{order =} \StringTok{"hclust"}\NormalTok{)}
\end{Highlighting}
\end{Shaded}

\includegraphics{test-book_files/figure-latex/unnamed-chunk-13-4.pdf}

\begin{Shaded}
\begin{Highlighting}[]
\NormalTok{corrplot}\OperatorTok{::}\KeywordTok{corrplot}\NormalTok{(M, }\DataTypeTok{method =} \StringTok{"color"}\NormalTok{, }\DataTypeTok{type =} \StringTok{"upper"}\NormalTok{, }\DataTypeTok{order =} \StringTok{"hclust"}\NormalTok{, }\DataTypeTok{addCoef.col =} \StringTok{"black"}\NormalTok{)}
\end{Highlighting}
\end{Shaded}

\includegraphics{test-book_files/figure-latex/unnamed-chunk-13-5.pdf}

\begin{Shaded}
\begin{Highlighting}[]
\NormalTok{corrplot}\OperatorTok{::}\KeywordTok{corrplot}\NormalTok{(M, }\DataTypeTok{method =} \StringTok{"color"}\NormalTok{, }\DataTypeTok{type =} \StringTok{"upper"}\NormalTok{, }\DataTypeTok{order =} \StringTok{"hclust"}\NormalTok{, }\DataTypeTok{addCoef.col =} \StringTok{"black"}\NormalTok{, }\DataTypeTok{tl.col=}\StringTok{"black"}\NormalTok{)}
\end{Highlighting}
\end{Shaded}

\includegraphics{test-book_files/figure-latex/unnamed-chunk-13-6.pdf}

\begin{Shaded}
\begin{Highlighting}[]
\NormalTok{corrplot}\OperatorTok{::}\KeywordTok{corrplot}\NormalTok{(M, }\DataTypeTok{method =} \StringTok{"color"}\NormalTok{, }\DataTypeTok{type =} \StringTok{"upper"}\NormalTok{, }\DataTypeTok{order =} \StringTok{"hclust"}\NormalTok{, }\DataTypeTok{addCoef.col =} \StringTok{"black"}\NormalTok{, }\DataTypeTok{tl.col=}\StringTok{"black"}\NormalTok{, }\DataTypeTok{tl.srt =} \DecValTok{45}\NormalTok{)}
\end{Highlighting}
\end{Shaded}

\includegraphics{test-book_files/figure-latex/unnamed-chunk-13-7.pdf}

\hypertarget{if_else-and-case_when-comparison}{%
\chapter{if\_else() and case\_when(): Comparison}\label{if_else-and-case_when-comparison}}

\hypertarget{case_when}{%
\section{case\_when()}\label{case_when}}

case\_when() from \url{https://www.rdocumentation.org/packages/dplyr/versions/0.7.8/topics/case_when}

\begin{Shaded}
\begin{Highlighting}[]
\NormalTok{x <-}\StringTok{ }\DecValTok{1}\OperatorTok{:}\DecValTok{50}
\NormalTok{y <-}\StringTok{ }\DecValTok{51}\OperatorTok{:}\DecValTok{100}

\NormalTok{df <-}\StringTok{ }\KeywordTok{data.frame}\NormalTok{(x,y)}
\NormalTok{df}
\end{Highlighting}
\end{Shaded}

\begin{verbatim}
##     x   y
## 1   1  51
## 2   2  52
## 3   3  53
## 4   4  54
## 5   5  55
## 6   6  56
## 7   7  57
## 8   8  58
## 9   9  59
## 10 10  60
## 11 11  61
## 12 12  62
## 13 13  63
## 14 14  64
## 15 15  65
## 16 16  66
## 17 17  67
## 18 18  68
## 19 19  69
## 20 20  70
## 21 21  71
## 22 22  72
## 23 23  73
## 24 24  74
## 25 25  75
## 26 26  76
## 27 27  77
## 28 28  78
## 29 29  79
## 30 30  80
## 31 31  81
## 32 32  82
## 33 33  83
## 34 34  84
## 35 35  85
## 36 36  86
## 37 37  87
## 38 38  88
## 39 39  89
## 40 40  90
## 41 41  91
## 42 42  92
## 43 43  93
## 44 44  94
## 45 45  95
## 46 46  96
## 47 47  97
## 48 48  98
## 49 49  99
## 50 50 100
\end{verbatim}

\begin{Shaded}
\begin{Highlighting}[]
\KeywordTok{case_when}\NormalTok{(}
\NormalTok{  x }\OperatorTok\StringTok{ }\DecValTok{35} \OperatorTok{==}\StringTok{ }\DecValTok{0} \OperatorTok{~}\StringTok{ "fizz buzz"}\NormalTok{,}
\NormalTok{  x }\OperatorTok\StringTok{ }\DecValTok{5} \OperatorTok{==}\StringTok{ }\DecValTok{0} \OperatorTok{~}\StringTok{ "fizz"}\NormalTok{,}
\NormalTok{  x }\OperatorTok\StringTok{ }\DecValTok{7} \OperatorTok{==}\StringTok{ }\DecValTok{0} \OperatorTok{~}\StringTok{ "buzz"}\NormalTok{,}
  \OtherTok{TRUE} \OperatorTok{~}\StringTok{ }\KeywordTok{as.character}\NormalTok{(x)}
\NormalTok{)}
\end{Highlighting}
\end{Shaded}

\begin{verbatim}
##  [1] "1"         "2"         "3"         "4"         "fizz"      "6"        
##  [7] "buzz"      "8"         "9"         "fizz"      "11"        "12"       
## [13] "13"        "buzz"      "fizz"      "16"        "17"        "18"       
## [19] "19"        "fizz"      "buzz"      "22"        "23"        "24"       
## [25] "fizz"      "26"        "27"        "buzz"      "29"        "fizz"     
## [31] "31"        "32"        "33"        "34"        "fizz buzz" "36"       
## [37] "37"        "38"        "39"        "fizz"      "41"        "buzz"     
## [43] "43"        "44"        "fizz"      "46"        "47"        "48"       
## [49] "buzz"      "fizz"
\end{verbatim}

\hypertarget{compare-this-with-if_else}{%
\section{Compare this with if\_else()}\label{compare-this-with-if_else}}

\begin{Shaded}
\begin{Highlighting}[]
\KeywordTok{if_else}\NormalTok{(x }\OperatorTok\StringTok{ }\DecValTok{2} \OperatorTok{==}\StringTok{ }\DecValTok{0}\NormalTok{, }\StringTok{"even"}\NormalTok{, }\StringTok{"odd"}\NormalTok{)}
\end{Highlighting}
\end{Shaded}

\begin{verbatim}
##  [1] "odd"  "even" "odd"  "even" "odd"  "even" "odd"  "even" "odd"  "even"
## [11] "odd"  "even" "odd"  "even" "odd"  "even" "odd"  "even" "odd"  "even"
## [21] "odd"  "even" "odd"  "even" "odd"  "even" "odd"  "even" "odd"  "even"
## [31] "odd"  "even" "odd"  "even" "odd"  "even" "odd"  "even" "odd"  "even"
## [41] "odd"  "even" "odd"  "even" "odd"  "even" "odd"  "even" "odd"  "even"
\end{verbatim}

\hypertarget{subset}{%
\chapter{Subsetting}\label{subset}}

From \url{https://www.r-bloggers.com/5-ways-to-subset-a-data-frame-in-r/}

Note: since this is down for maintenance, I will turn off evaluation on these chunks:

\begin{Shaded}
\begin{Highlighting}[]
\NormalTok{education <-}\StringTok{ }\KeywordTok{read.csv}\NormalTok{(}\StringTok{"https://vincentarelbundock.github.io/Rdatasets/csv/robustbase/education.csv"}\NormalTok{, }\DataTypeTok{stringsAsFactors =} \OtherTok{FALSE}\NormalTok{)}

\KeywordTok{colnames}\NormalTok{(education) <-}\StringTok{ }\KeywordTok{c}\NormalTok{(}\StringTok{"X"}\NormalTok{,}\StringTok{"State"}\NormalTok{,}\StringTok{"Region"}\NormalTok{,}\StringTok{"Urban.Population"}\NormalTok{,}\StringTok{"Per.Capita.Income"}\NormalTok{,}\StringTok{"Minor.Population"}\NormalTok{,}\StringTok{"Education.Expenditures"}\NormalTok{)}

\KeywordTok{glimpse}\NormalTok{(education)}
\end{Highlighting}
\end{Shaded}

\begin{verbatim}
## Rows: 50
## Columns: 7
## $ X                      <int> 1, 2, 3, 4, 5, 6, 7, 8, 9, 10, 11, 12, 13, 1...
## $ State                  <chr> "ME", "NH", "VT", "MA", "RI", "CT", "NY", "N...
## $ Region                 <int> 1, 1, 1, 1, 1, 1, 1, 1, 1, 2, 2, 2, 2, 2, 2,...
## $ Urban.Population       <int> 508, 564, 322, 846, 871, 774, 856, 889, 715,...
## $ Per.Capita.Income      <int> 3944, 4578, 4011, 5233, 4780, 5889, 5663, 57...
## $ Minor.Population       <int> 325, 323, 328, 305, 303, 307, 301, 310, 300,...
## $ Education.Expenditures <int> 235, 231, 270, 261, 300, 317, 387, 285, 300,...
\end{verbatim}

\hypertarget{subsetting-using-brackets}{%
\section{Subsetting using brackets}\label{subsetting-using-brackets}}

\begin{Shaded}
\begin{Highlighting}[]
\NormalTok{education[}\KeywordTok{c}\NormalTok{(}\DecValTok{10}\OperatorTok{:}\DecValTok{21}\NormalTok{),}\KeywordTok{c}\NormalTok{(}\DecValTok{2}\NormalTok{,}\DecValTok{6}\OperatorTok{:}\DecValTok{7}\NormalTok{)]}
\end{Highlighting}
\end{Shaded}

\begin{verbatim}
##    State Minor.Population Education.Expenditures
## 10    OH              324                    221
## 11    IN              329                    264
## 12    IL              320                    308
## 13    MI              337                    379
## 14    WI              328                    342
## 15    MN              330                    378
## 16    IA              318                    232
## 17    MO              309                    231
## 18    ND              333                    246
## 19    SD              330                    230
## 20    NB              318                    268
## 21    KS              304                    337
\end{verbatim}

\hypertarget{subset-using-brackets-by-omitting-the-rows-and-columns-we-dont-want}{%
\section{Subset using brackets by omitting the rows and columns we don't want}\label{subset-using-brackets-by-omitting-the-rows-and-columns-we-dont-want}}

\begin{Shaded}
\begin{Highlighting}[]
\NormalTok{education[}\OperatorTok{-}\KeywordTok{c}\NormalTok{(}\DecValTok{1}\OperatorTok{:}\DecValTok{9}\NormalTok{,}\DecValTok{22}\OperatorTok{:}\DecValTok{50}\NormalTok{),}\OperatorTok{-}\KeywordTok{c}\NormalTok{(}\DecValTok{1}\NormalTok{,}\DecValTok{3}\OperatorTok{:}\DecValTok{5}\NormalTok{)]}
\end{Highlighting}
\end{Shaded}

\begin{verbatim}
##    State Minor.Population Education.Expenditures
## 10    OH              324                    221
## 11    IN              329                    264
## 12    IL              320                    308
## 13    MI              337                    379
## 14    WI              328                    342
## 15    MN              330                    378
## 16    IA              318                    232
## 17    MO              309                    231
## 18    ND              333                    246
## 19    SD              330                    230
## 20    NB              318                    268
## 21    KS              304                    337
\end{verbatim}

\hypertarget{subset-using-brackets-in-combination-with-the-which-function-and-the-in-operator}{%
\section{Subset using brackets in combination with the which() function and the \%in\% operator}\label{subset-using-brackets-in-combination-with-the-which-function-and-the-in-operator}}

\begin{Shaded}
\begin{Highlighting}[]
\NormalTok{education[}\KeywordTok{which}\NormalTok{(education}\OperatorTok{$}\NormalTok{Region }\OperatorTok{==}\StringTok{ }\DecValTok{2}\NormalTok{),}\KeywordTok{names}\NormalTok{(education) }\OperatorTok\StringTok{ }\KeywordTok{c}\NormalTok{(}\StringTok{"State"}\NormalTok{,}\StringTok{"Minor.Population"}\NormalTok{,}\StringTok{"Education.Expenditures"}\NormalTok{)]}
\end{Highlighting}
\end{Shaded}

\begin{verbatim}
##    State Minor.Population Education.Expenditures
## 10    OH              324                    221
## 11    IN              329                    264
## 12    IL              320                    308
## 13    MI              337                    379
## 14    WI              328                    342
## 15    MN              330                    378
## 16    IA              318                    232
## 17    MO              309                    231
## 18    ND              333                    246
## 19    SD              330                    230
## 20    NB              318                    268
## 21    KS              304                    337
\end{verbatim}

\hypertarget{subset-using-the-subset-function}{%
\section{Subset using the subset() function}\label{subset-using-the-subset-function}}

\begin{Shaded}
\begin{Highlighting}[]
\KeywordTok{subset}\NormalTok{(education, Region }\OperatorTok{==}\StringTok{ }\DecValTok{2}\NormalTok{, }\DataTypeTok{select =} \KeywordTok{c}\NormalTok{(}\StringTok{"State"}\NormalTok{,}\StringTok{"Minor.Population"}\NormalTok{,}\StringTok{"Education.Expenditures"}\NormalTok{))}
\end{Highlighting}
\end{Shaded}

\begin{verbatim}
##    State Minor.Population Education.Expenditures
## 10    OH              324                    221
## 11    IN              329                    264
## 12    IL              320                    308
## 13    MI              337                    379
## 14    WI              328                    342
## 15    MN              330                    378
## 16    IA              318                    232
## 17    MO              309                    231
## 18    ND              333                    246
## 19    SD              330                    230
## 20    NB              318                    268
## 21    KS              304                    337
\end{verbatim}

\hypertarget{subset-using-dyplyrs-filter-and-select}{%
\section{Subset using dyplyr's filter() and select()}\label{subset-using-dyplyrs-filter-and-select}}

\begin{Shaded}
\begin{Highlighting}[]
\KeywordTok{select}\NormalTok{(}\KeywordTok{filter}\NormalTok{(education, Region }\OperatorTok{==}\StringTok{ }\DecValTok{2}\NormalTok{),}\KeywordTok{c}\NormalTok{(State,Minor.Population}\OperatorTok{:}\NormalTok{Education.Expenditures))}
\end{Highlighting}
\end{Shaded}

\begin{verbatim}
##    State Minor.Population Education.Expenditures
## 1     OH              324                    221
## 2     IN              329                    264
## 3     IL              320                    308
## 4     MI              337                    379
## 5     WI              328                    342
## 6     MN              330                    378
## 7     IA              318                    232
## 8     MO              309                    231
## 9     ND              333                    246
## 10    SD              330                    230
## 11    NB              318                    268
## 12    KS              304                    337
\end{verbatim}

\bibliography{book.bib,packages.bib}

\end{document}
